\chapter{Conclusions and perspectives}
\label{chapConclusions}

In this thesis, we have built better GP regression models by integrating the prior knowledge of aircraft design with experimental or simulation data. This task is very pertinent because of the following reasons:

\begin{enumerate}
\item \textbf{Cost}: Generating highly accurate data of physical systems is a costly exercise. We can use the surrogate models as a cheap substitute for accurate models.
\item \textbf{Prior knowledge}: In the past 100 years several types of relationships have already been discovered for aircraft design tasks. We would like to reuse them in our model building. 
\item \textbf{Physically consistent}: Enforcing physical laws of the system will give rise to robust and physically consistent models.
\end{enumerate}

We demonstrate how to incorporate the prior information by mainly changing the covariance functions of the GP. Prior information in terms of smoothness, linearity, differentiability, etc. can be easily encoded using simple covariance functions. Similarly, relationships between multiple outputs can be learned or enforced by treating the `output number' as an extra dimension, this enables us again to easily create covariance functions for `Multi-task GPs'. We now come back to the four questions that were posed in the first chapter of this thesis and highlight the contributions for each topic.

\subsubsection*{How to add \textit{a priori} information of a pattern in a learning algorithm?}
Several types of prior information of the pattern can be enforced into a GP regression algorithm. We first,  demonstrate how to use GP regression to automatically detect modal parameters of structural dynamics. Using the spectral mixture kernels, peaks in the frequency domain can be identified automatically. To the best of our knowledge, such a method has not been used in the earlier literature to identify modal parameters. We then, demonstrate how to identify the onset of non-linear behaviour in physical systems by using a change-point kernel. For example we identify the initiation of flow separation in NACA 0012 airfoil and initiation of plasticity in AL6061 alloy using a statistical criteria for automatic detection of non-linearity \cite{chiplunkar:hal-01555401}. We finally build a GP model to predict the position of aerodynamic shock in the transonic regime (section \ref{subecInterpolationOfAerodynamicPressures}) \cite{oatao18004}. The predicted position and accuracy of shock is better than the state-of the art models. 
\subsubsection*{How to merge \textit{a priori} information of simulations with experiments?}
Multi-fidelity models in GPs were first introduced by \cite{kennedy2000predicting} and have been popular in making cost effective surrogate modelling of simulation codes. We demonstrate how multi-fidelity GP models can be extended to perform model updating or extrapolation of experimental data. We include an error term and a translation term into a normal multi-fidelity model to account for differences in the simulation model and experimental data. 

\subsubsection*{How to add \textit{a priori} information of relationships between measurements?}
The physical laws or relationships have been developed after centuries of observations and iterative experiments. We extend the framework of gradient enhanced kriging to enforce any relationship between outputs. We also provide a novel graphical interpretation of this enforced multiple-output regression process. The improvement in model accuracy is then validated on both on a toy data-set and a data-set of real flight-test.

\subsubsection*{How to scale GP regression to large data sets?}
Calculating the precision matrix is an important task in choosing appropriate hyper-parameters and calculating the posterior mean and variance. Unfortunately, this task puts an upper limit of $N \sim 10^4$ on the number of data points. This problem is further aggravated when we are creating a joint Gram matrix between several outputs. We demonstrate how to scale single output GPs using `Sparse GPs'  and `Distributed GPs' methods.

We demonstrate the capability of scaling, by building a distributed GP model for interpolating pressures on millions of nodes in a CFD mesh. This surrogate model was used in a recent Airbus flight test campaign to compare pressures predicted from a high-fidelity CFD computation to pressures measured on the wing, in real time. We then demonstrate how to scale up `Multiple-output GPs' to large number of data points, both using a sparse approximation and a distributed GP approximation. We then validate the scalability on a toy data-set and a data-set from flight test and compare the accuracy of distributed GP and variational inference on the toy-data-set. 


\section{Perspectives}
Several future paths can be explored by merging prior knowledge of aircraft design with GP regression. The developments in thesis for the field of aircraft engineering can be easily replicated to other engineering domains. Other domains also exhibit a similar problematic, where significant gains can be achieved by combining data and prior knowledge \cite{lookman2017statistical, jidling2017linearly, journals/jmlr/AlvarezLL09}. The proposed GP models can then be used for \textbf{system identification} or \textbf{system control} \cite{kocijan2016modelling, frigola2014variational, deisenroth2011pilco}. 


Similarly, by incorporating prior patterns we can create more \textbf{accurate surrogate models} when compared to state of the art techniques, these surrogate models can be then used as cheap replacements to more costly simulation codes. An upcoming field of research called Probabilistic Numerics, uses similar Bayesian principles to propagate uncertainty through engineering models \cite{2014arXiv14022058H}. 

The capability to integrate multiple outputs correlated through a prior known relationship opens other options for future research. Merging simulation model with experimental data is recurring theme in several verification and validation problems. Multi-output GPs provide a different way to merge these two outputs, leveraging the prior knowledge of simulation model in presence of \textbf{non-separable} differences should be studied in detail. Moreover, by enforcing physical laws in a surrogate model, we can develop schemes to perform \textbf{cheap MDO}, automatically \textbf{detect faulty sensors} and enforce \textbf{inequality constraints}. 

