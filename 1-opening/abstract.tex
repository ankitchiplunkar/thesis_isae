%\begin{vcenterpage}
\noindent\rule[2pt]{\textwidth}{0.5pt}
%\begin{center}
%{\large\textbf{Title in english\\}}
%\end{center}
{\large\textbf{Abstract ---}}
Due to the high cost of performing experiments on physical systems, numerical models become an efficient means to designing them. Traditionally, these models were built by iteratively performing experiments in a controlled environment. A more cost effective method of building models is by using machine learning algorithms, which infer patterns from data and can be used to perform interpolation and extrapolation. In this thesis, we propose to build better Gaussian Process (GP) regression models by integrating the prior knowledge of Aircraft design with experimental data. 

We demonstrate how to incorporate the prior information by mainly changing the covariance functions of the GP. Prior information in terms of smoothness, linearity, differentiability, etc. can be easily encoded using simple covariance functions. Using spectral mixture kernels we demonstrate how to build models for structural dynamic experiments and automatically identify dynamic parameters such as modal frequency. To the best of our knowledge, such a method has not been used in earlier literature to identify the modal parameters. We then use the change-point kernels to identify onset of non-linearity, and use a neural network kernel to interpolate shock in transonic regime. For each application we compare the proposed model with the state-of-the-art model and demonstrate the cost or performance gains achieved. 

Similarly, relationships between multiple outputs can be learned or enforced by manipulating the covariance functions. We first demonstrate how the prior information of simulation models can be incorporated to perform extrapolation of aircraft loads using the flight-test data. We then incorporate the physical laws of flight-mechanics into GP regression to identify flight loads. For both use-cases, we first validate the methodology on a synthetic dataset and then demonstrate the gains on a flight-test dataset.

A limitation of GPs is that they scale very poorly with data. There are two main methods of scaling GPs. The first reduces the dataset using a limited set of inducing inputs, while the second distributes the dataset into smaller batches. We demonstrate the scaling capabilities achieved, by building a GP model for interpolating pressures on millions of nodes in a CFD mesh. To the best of our knowledge this scale of GP model has not been created before for interpolating aerodynamic pressures. A similar surrogate model was used in a recent Airbus flight test campaign to compare the pressures predicted from high-fidelity CFD computations to the pressures measured real time on the wing. We then demonstrate how to scale up Multi-Task GPs to large number of data points, using both approximations. The proposed approach was then validated on a synthetic dataset and on a flight test dataset.

{\large\textbf{Keywords:}}
    Gaussian Process, Flight-Mechanics, Aircraft design, Structural Dynamics, Shock Interpolation.
\\
\noindent\rule[2pt]{\textwidth}{0.5pt}

\pagebreak

\noindent\rule[2pt]{\textwidth}{0.5pt}

{\large\textbf{Résumé ---}}
En raison du coût élevé des essais sur les systèmes physiques, les modèles numériques deviennent un moyen préférentiel pour les concevoir. Une méthode plus efficiente de construction de modèles consiste à utiliser des algorithmes d'apprentissage statistique, qui déduisent des modèles à partir de données et peuvent être utilisés pour interpoler et extrapoler. Dans cette thèse, nous proposons de construire de meilleurs modèles de régression de Processus Gaussiens (GPs) en intégrant la connaissance préalable de la conception d'aéronef avec des données expérimentales.

Nous démontrons comment intégrer l'information préalable en modifiant principalement les fonctions de covariance du GP. En utilisant les noyaux de mélange spectral, nous démontrons comment construire des modèles pour des expériences dynamiques structurelles et identifier automatiquement des paramètres dynamiques comme la fréquence modale. Nous utilisons ensuite le noyau de `change-point' pour identifier l'apparition des non-linéarités et utilisons un noyau de réseau neuronal pour interpoler le choc dans le régime transsonique. Pour chaque application, nous comparons le modèle proposé avec le modèle d'état de l'art et démontrons les gains de coût ou de performance obtenus.


De même, les relations entre plusieurs sorties peuvent être apprises ou imposées en manipulant les fonctions de covariance. Nous démontrons d'abord comment l'information préalable du modèle de simulation peut être incorporée pour effectuer une extrapolation des données expérimentales. Nous incorporons ensuite les lois physiques de la mécanique de vol dans la régression du GP pour identifier les charges de vol. Pour les deux cas d'utilisation, nous validons d'abord la méthodologie sur un ensemble de données synthétiques, puis démontrons les gains sur des données d'essai en vol.

Effectuer une régression GP pour un grand nombre de données devient coûteux très vite, deux méthodes pour étendre des GPs existent. La première 'Sparse GPs' utilise un ensemble des points inductifs pour réduire le coût de calcul de la matrice de précision. La seconde est appelée `Distributed GPs' elle divise l'ensemble de données en sous-ensembles plus petits, en divisant le modèle en plusieurs lots. Nous démontrons l’avantage de, construit un modèle GP pour interpoler les pressions sur des millions de nœuds dans un maillage CFD. À notre connaissance, tel modèle GP n'a jamais été mis en œuvre pour l'interpolation des pressions. Un modèle de substitution similaire a été utilisé pour une récente campagne d'essais en vol d'Airbus. Nous démontrons ensuite comment étendre les `Multi-Task GPs' (MTGP) à un grand nombre de sorties, à la fois en utilisant une approximation `Sparse GPs’ et une approximation de `Distributed GPs'. L'approche proposée a ensuite été validée sur un ensemble de données synthétiques et sur un ensemble de données d'essai en vol.

{\large\textbf{Mots clés :}}
    Processus gaussien, Mécanique de vol, Conception d'aéronefs, Dynamique structurale, Interpolation de choc.
\\
\noindent\rule[2pt]{\textwidth}{0.5pt}





%\begin{center}
%  Institut Cl\'ement Ader, 3 Rue Caroline Aigle\\
%  Toulouse, France
%\end{center}
%\end{vcenterpage}

%%% Local Variables: 
%%% mode: latex
%%% TeX-master: "../phdthesis"
%%% End: 
