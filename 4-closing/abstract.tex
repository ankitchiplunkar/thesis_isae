\begin{vcenterpage}

\noindent\rule[2pt]{\textwidth}{0.5pt}

{\large\textbf{Résumé ---}}
   Dans cette thèse, nous proposons de construire de meilleurs modèles de régression GP en intégrant les connaissances antérieures sur la conception d'Aéronef avec des données expérimentales. En raison du haut coût d'exécuter des expériences sur des systèmes physiques, les concevoir purement par l'expérimentation devient un exercice coûteux. Ainsi, les modèles deviennent un moyen évident de concevoir des systèmes physiques. Traditionnellement, ces modèles ont été construits par des instituts de recherche, en effectuant itérativement des expériences dans un environnement contrôlé. Une méthode plus rentable de construire des modèles est en utilisant des algorithmes d'apprentissage automatique, qui déduisent des modèles de données et peuvent être utilisés pour effectuer l'interpolation et l'extrapolation. Tout au long de cette thèse, nous fournissons des méthodes pour incorporer des informations \textit{apriori} pour construire de meilleurs modèles de systèmes physiques. Nous démontrons d'abord comment intégrer l'information \textit{apriori} d'un forme en construisant des modèles pour détecter les paramètres dynamiques structurels, détecter l'apparition de la plasticité sur un alliage d'aluminium et interpoler le choc sur une aile en régime transonique. On deuxièmement, démontrons comment intégrer les informations \textit{apriori} des relations entre les résultats et la construction de modèles GP. Nous utilisons cette méthodologie pour extrapoler des données expérimentales en utilisant des données de simulation et construire des modèles cohérents de mécanique de vol. La dernier contribution de cette thèse est que nous fournissons des moyens pour mettre en échec les GP multi-output. Les modèles GP sont difficiles à mettre à l'échelle pour un grand nombre de points de données, ces limites deviennent plus restrictives lorsque nous abordons plusieurs sorties. Pour chaque application, nous comparons le modèle proposé avec le modèle de pointe et démontrons les gains de coût ou de performance obtenus. Nous validons d'abord la méthodologie sur un ensemble de données de jouets synthétiques et démontrons les gains sur un ensemble de données expérimental réel.

{\large\textbf{Mots clés :}}
    Processus gaussien, Mécanique de vol, Conception d'aéronefs, Dynamique structurale, Interpolation de choc.
\\
\noindent\rule[2pt]{\textwidth}{0.5pt}

\vspace{0.5cm}

\noindent\rule[2pt]{\textwidth}{0.5pt}
%\begin{center}
%{\large\textbf{Title in english\\}}
%\end{center}
{\large\textbf{Abstract ---}}
    In this thesis, we propose to build better GP regression models by integrating the prior knowledge of Aircraft design with experimental data. Due to the high cost of performing experiments on physical systems, designing them purely through experimentation becomes a costly exercise. Thus models become an obvious means to designing physical systems. Traditionally, these models were built by research institutes, through iteratively performing experiments in a controlled environment. A more cost effective method of building models is by using machine learning algorithms, which infer patterns from data and can be used to perform interpolation and extrapolation. Throughout this thesis we provide methods to incorporate \textit{apriori} information to build better models of the physical systems. We first demonstrate how to incorporate information of an \textit{apriori} pattern by building models to detect structural dynamic parameters, detect onset of plasticity on an Aluminum alloy, and interpolate shock on a wing in transonic regime. We secondly, demonstrate how to incorporate \textit{apriori} information of relationships between outputs to building GP models. We use this methodology to extrapolate experimental data using simulation data and build consistent flight-mechanics models. The final contribution of this thesis is that we provide means to scale multi-output GPs. The GP models are difficult to scale for large number of data-points, this limits becomes more restrictive when we are tackling multiple outputs. 
    For each application we compare the proposed model with the state of the art model and demonstrate the cost or performance gains achieved. We first validate the methodology on a synthetic toy-dataset and then demonstrate the gains on a real experimental dataset. 

{\large\textbf{Keywords:}}
    Gaussian Process, Flight-Mechanics, Aircraft design, Structural Dynamics, Shock Interpolation.
\\
\noindent\rule[2pt]{\textwidth}{0.5pt}
\begin{center}
  Institut Cl\'ement Ader, 3 Rue Caroline Aigle\\
  Toulouse, France
\end{center}
\end{vcenterpage}

%%% Local Variables: 
%%% mode: latex
%%% TeX-master: "../phdthesis"
%%% End: 
