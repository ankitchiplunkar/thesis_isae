\begin{vcenterpage}

\noindent\rule[2pt]{\textwidth}{0.5pt}

{\large\textbf{Résumé ---}}
Dans cette thèse, nous proposons de construire de meilleurs modèles GP en intégrant les connaissances antérieures avec des données expérimentales. En raison du coût élevé de l'exécution d'expériences sur les systèmes physiques, les modèles deviennent un moyen évident de concevoir des systèmes physiques. Traditionnellement, ces modèles ont été construits par des instituts de recherche; une méthode plus rentable de construction de modèles consiste à utiliser des algorithmes d'apprentissage automatique. Nous démontrons comment créer des modèles en intégrant l'information antérieures en modifiant les fonctions de covariance. Nous proposons des modèles GP pour détecter l'apparition de la non-linéarité, détecter les paramètres modaux et interpoler la position du choc. De même, les lois physiques entre plusieurs sorties peuvent être appliquées en manipulant les fonctions de covariance. Pour chaque application, nous comparons le modèle proposé avec le modèle d'état de l'art et démontrons les gains de coût ou de performance obtenus.

{\large\textbf{Mots clés :}}
    Processus gaussien, Mécanique de vol, Conception d'aéronefs, Dynamique structurale, Interpolation de choc.
\\
\noindent\rule[2pt]{\textwidth}{0.5pt}

\vspace{0.5cm}

\noindent\rule[2pt]{\textwidth}{0.5pt}
%\begin{center}
%{\large\textbf{Title in english\\}}
%\end{center}
{\large\textbf{Abstract ---}}
In this thesis, we propose to build better GP models by integrating the prior knowledge of Aircraft design with experimental data. Due to the high cost of performing experiments on physical systems, models become an obvious means to designing physical systems. Traditionally, these models were built by research institutes; a more cost effective method of building models is by using machine learning algorithms. We demonstrate how to create models by incorporating the prior information by changing the covariance functions of the GP,. We propose GP models to detect onset of non-linearity, detect modal parameters and interpolate position of shock. Similarly, physical laws between multiple outputs can be enforced by manipulating the covariance functions, we propose to integrate flight-mechanics to better identify loads using these models. For each application we compare the proposed model with the state of the art model and demonstrate the cost or performance gains achieved. 

{\large\textbf{Keywords:}}
    Gaussian Process, Flight-Mechanics, Aircraft design, Structural Dynamics, Shock Interpolation.
\\
\noindent\rule[2pt]{\textwidth}{0.5pt}
\begin{center}
  Institut Cl\'ement Ader, 3 Rue Caroline Aigle\\
  Toulouse, France
\end{center}
\end{vcenterpage}

%%% Local Variables: 
%%% mode: latex
%%% TeX-master: "../phdthesis"
%%% End: 